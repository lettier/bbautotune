\chapter{Conclusion}

\label{Chapter6}

%\todo{Write this last.}

\section{Genetic Algorithms}

The GA outlined in this thesis employed a wide variety of techniques presented by previous works. Unfortunately, there is no unanimously recognized theory of GAs \cite{Beasley93anoverview}. Furthermore, there is no proven optimal set of GA parameters for any given problem being solved by a GA \cite{ColinReeves}. There are, however, a few schools of thought and generally accepted guidelines when developing a GA \cite{ColinReeves}\cite{Beasley93anoverview}. As to why GAs work, at least for BCGAs, a few ideas have been presented such as John Holland's schema theorem and David Goldberg's building block hypothesis \cite{Beasley93anoverview}. 

\section{SimPL}

%\todo{Change the wording to reflect that it \textit{was} helpful to do SimPL since BBAutoTune was successful.}

The EA developed for SimPL proved to be a robust basis for the EA needed to solve the harder problem of tuning a 3D physics engine (project BBAutoTune). The principles and techniques of evolutionary algorithms learned during the SimPL project certainly carried over to the more difficult project, BBAutoTune. And while the problem domain of SimPL and BBAutoTune were only somewhat similar, the problems faced and worked out during the development of SimPL alleviated problems faced while developing BBAutoTune. As the results show, the EA for SimPL performed well, producing neural network weight solutions that had the paddle keeping the ball in the arena for almost a minute. Had it not been for the round termination criteria of the ball's velocity magnitude dropping below $100$, most of the paddles (with high fitnesses) would have kept the ball in the arena indefinitely. Thus, the goal to learn about and to cultivate an EA capable of tuning parameters with respect to a fitness landscape was certainly accomplished. 

\section{BBAutoTune}

Using the real robot forward motion data, BBAutoTune was consistently able to tune the physics engine such that the reality gap between simulation and reality was extremely small. For all runs of the GA, BBAutoTune was able to find nearly optimal physics parameters in no more than five hours. This is particularly impressive considering the search space size was $(2^{53})^{11}=2^{53*11}=2^{583}=3.165829139\times10^{175}$ possible states\footnote{On a 64-bit computer architecture, there is approximately $2^{53}$ representable (double precision) floats between $0.0$ and $1.0$ \cite{Patterson:2013:COD:2568134}\cite{PythonFloat}. Each genome in BBAutoTune contained an array of 11 floats which represented the 11 possible tunable physics parameters. The range of each float in the array was $[0.0,1.0]$.}. It is even more impressive considering the many days lost trying to find reasonable parameters by hand (during preliminary work) only to abandon the physics engine altogether due to consistent instability issues. Interestingly, the physics parameters found were not necessarily intuitive nor did they coincide with their real world counterparts. For example, in experiment two, gravity was $\sim2.89\frac{m}{s^2}$ versus earth's gravitational constant $\sim9.81\frac{m}{s^2}$ and the collision bound type was sphere versus the cylindrical shape of the robot's wheels.   

\section{BlenderSim}

Interpreting the plot of the simulated versus real robot motion along with the Fr{\'e}chet and Hausdorff distances, one can see that the simulated motion was very close to the real robot motion. Considering the largest move the (simulated or real) robot can make at any one time on the discrete arena grid is $\sim32.52cm$\footnote{Overlaid on the robot arena is a grid spaced $23cm\times23cm$ for both the simulated and real arenas. This discrete grid is used to compute the A* paths that take the robots from their starting positions to the task points in the arena. Moving diagonally from one grid square to another requires a distance of $\sqrt{23^2cm+23^2cm}=32.526911935cm$.} and that the largest distance between the simulated and real robots paths was a Fr{\'e}chet distance of $\sim40.57cm$, the thesis was certainly demonstrated and its hypothesis was supported.

\section{Future Work}

Future work, concerning BBAutoTune and BlenderSim, could involve the following:
% \begin{enumerate}
%  \item modify the motion capture process so that different motion commands (other than \textit{forward} $25cm$) are given to the robot, including turning, which may provide different learned physics parameter settings;
%  \item run new (or logged) experiments in both the physical environment and in BlenderSim, comparing performance metrics to see how similar the complete experimental runs are; and
%  \item learn physics engine motion models for other types of robots in BBAutoTune, and try putting those into BlenderSim;
% \end{enumerate}
\begin{enumerate}
 \item capture additional real-robot motion data on the SRV-1 by issuing it different commands (such as turning) instead of just forward;
 \begin{enumerate}
  \item take this additional motion data and in BBAutoTune, learn the physics parameters needed to simulate these different robot commands in BlenderSim;
 \end{enumerate}
 \item run new (or other previously logged) experiments in tandem between HRTeam and BlenderSim, making further comparisons between simulated and real robot paths; and
 \item in BBAutoTune, learn the physics parameters necessary to simulate the motion of other robots (other than the SRV-1).
\end{enumerate}


